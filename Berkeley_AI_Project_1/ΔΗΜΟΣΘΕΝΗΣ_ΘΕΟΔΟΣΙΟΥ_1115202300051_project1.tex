\documentclass{article}
\usepackage[utf8]{inputenc}
\usepackage[greek]{babel}
\thispagestyle{empty} \textwidth=16.5cm \textheight=28cm
\oddsidemargin=0cm \topmargin=-3.3cm
\usepackage{amsmath}

\title{\bfseries\small
\begin{flushleft}
ΕΘΝΙΚΟ ΚΑΙ ΚΑΠΟΔΙΣΤΡΙΑΚΟ ΠΑΝΕΠΙΣΤΗΜΙΟ ΑΘΗΝΩΝ\\
ΤΜΗΜΑ ΠΛΗΡΟΦΟΡΙΚΗΣ ΚΑΙ ΤΗΛΕΠΙΚΟΙΝΩΝΙΩΝ\\
ΜΑΘΗΜΑ: ΤΕΧΝΗΤΗ ΝΟΗΜΟΣΥΝΗ I\\
\end{flushleft}
}


\author{\Large Εργασία 1 \\ \\Δημοσθένης Θεοδοσίου (1115202300051)}
\date{Οκτώβριος 2024}

\begin{document}


\maketitle
\section*{Πρόβλημα 2:}
Έχετε το εξής πρόβλημα αναζήτησης:
\begin{itemize}
    \item Ο χώρος καταστάσεων αναπαρίσταται ως δένδρο.
    \item Ο κόμβος της ρίζας (αρχική κατάσταση) έχει τρεις κόμβους-παιδιά.
    \item Κάθε ένας από αυτούς τους κόμβους-παιδιά έχει επίσης τρεις κόμβους-παιδιά κ.ο.κ. Δηλαδή, το δένδρο έχει ομοιόμορφο παράγοντα διακλάδωσης ίσο με 3.
    \item Ο στόχος βρίσκεται στο βάθος 4.
\end{itemize}

Να υπολογίσετε θεωρητικά τον μικρότερο και το μεγαλύτερο αριθμό κόμβων που επεκτείνονται από κάθε έναν από τους παρακάτω αλγόριθμους αναζήτησης, υποθέτοντας ότι εκτελούν πλήρη αναζήτηση (δηλαδή, μέχρι να βρεθεί ο στόχος):
\begin{itemize}
    \item Αναζήτηση πρώτα κατά πλάτος (\textlatin{BFS})
    \item Αναζήτηση πρώτα κατά βάθος (\textlatin{DFS}). Υποθέστε ότι ο \textlatin{DFS} εξετάζει πάντα πρώτα το αριστερότερο παιδί.
    \item Αναζήτηση με επαναληπτική εκβάθυνση (\textlatin{IDS}).
\end{itemize}


\section*{Απάντηση:}
Στο πρόβλημα δεν διευκρινίζεται εάν ο χώρος καταστάσεων είναι πεπερασμένος ή άπειρος. Στην λύση μου θα εξετάσω όλες τις δυνατές περιπτώσεις. Επίσης θεωρώ πως η αρίθμιση των επιπέδων ξεκινάει από το 0.
Από τα δεδομένα του προβλήματος προκύπτει πως \textlatin{b} = 3 και \textlatin{d} = 4. 

\begin{document}

\begin{itemize}
    \item Αναζήτηση πρώτα κατά πλάτος (\textlatin{BFS})
\end{itemize}

Γενικά, ο \textlatin{BFS} σε γράφους έχει πολυπλοκότητα \(O(b^{(d+1)})\), αλλά θα θεωρήσουμε πως έχουμε την βελτιωμένη έκδοση με μικρότερη πολυπλοκότητα \(O(b^d)\). Η καλύτερη περίπτωση είναι ο κόμβος στόχος να βρίσκεται στο αριστερότερο σημείο του επιπέδου με βάθος 4. Αυτό προκύπτει αν αναπτύξουμε όλα τα επίπεδα μέχρι το 3 και αναπτύξουμε τον πρώτο κόμβο του τέταρτου επιπέδου. Επειδή κάθε επίπεδο έχει \(O(b^{(\alpha/\alpha\})\), έχουμε συνολικά 1+3+9+27+1=41 κόμβους. Τώρα η χειρότερη περίπτωση είναι ο κόμβος στόχος να βρίσκεται στο δεξιότερο σημείο του τέταρτου επιπέδου. Έτσι προκύπτουν 1+3+9+27+81=121 κόμβοι. Βλέπουμε πως στο τρέχον πρόβλημα δεν έχει επιρροή στον \textlatin{BFS} η περατότητα ή μη του χώρου καταστάσεων.

\begin{itemize}
    \item Αναζήτηση πρώτα κατά βάθος (\textlatin{DFS})
    \\  \\Γενικά ο αλγόριθμος \textlatin{DFS} έχει πολυπλοκότητα \(O(b^\mu)\).
    \begin{itemize}
        \item Περίπτωση 1 : Μη πεπερασμένος χώρος καταστάσεων
        \\ \\ Στην περίπτωση που ο χώρος καταστάσεων δεν είναι πεπερασμένος και ο \textlatin{DFS} εξετάζει πάντα το αριστερότερο παιδί, έχουμε δύο ενδεχόμενα. Εάν ο κόμβος στόχος είναι το αριστερότερο παιδί του 4ου επιπέδου, ο \textlatin{DFS} θα χρειαστεί να αναπτύξει μόλις 5 κόμβους. Σε κάθε άλλη περίπτωση όπου ο κόμβος στόχος δεν είναι το αριστερότερο στοιχείο του επιπέδου, ο αλγόριθμος θα αναπτύξει άπειρους κόμβους.
    \end{itemize}

    \begin{itemize}
        \item Περίπτωση 2 : Πεπερασμένος χώρος καταστάσεων
        \\ \\ Στην περίπτωση που ο χώρος καταστάσεων είναι πεπερασμένος και ο \textlatin{DFS} εξετάζει πάντα το αριστερότερο παιδί, έχουμε δύο ενδεχόμενα. Εάν ο κόμβος στόχος είναι το αριστερότερο παιδί του 4ου επιπέδου, ο \textlatin{DFS} θα χρειαστεί να αναπτύξει μόλις 5 κόμβους. Αυτή είναι η καλύτερη περίπτωση. Η χειρότερη περίπτωση είναι όταν ο κόμβος βρίσκεται πάλι στο δεξιότερο σημείο του τέταρτου επιπέδου.
        Εκεί δεν θα αναπτυχθούν μόνο 1 + 3 + 9 + 27 = 40 κόμβοι, όπως θα γινόταν εάν όλοι οι κόμβοι πήγαιναν μέχρι και το 4ο επίπεδο, αλλά στην πραγματικότητα θα αναπτυχθούν 40 + x κόμβοι. Το x εδώ αναπαριστά το άθροισμα των μεγεθών όλων εκείνων των υποδέντρων τα οποία έχουν ρίζα όλους τους κόμβους του 4ου επιπέδου πλην του τελευταίου. 
    \end{itemize}
    \end{itemize}

\begin{itemize}
    \item Αναζήτηση με επαναληπτική εκβάθυνση (\textlatin{IDS}) \\ \\
Εφόσον ο \textlatin{IDS} είναι μία σύζευξη του \textlatin{DFS} και του \textlatin{BFS}, είναι ασφαλές να υποθέσουμε ότι για κάθε επίπεδο που εξετάζει ο \textlatin{IDS} το κάνει εξετάζοντας πάντοτε το αριστερότερο παιδί πρώτα.
       Πάλι εδώ, η καλύτερη περίπτωση είναι το αριστερότερο παιδί του 4ου επιπέδου. Εδώ θα χρειαστεί ο \textlatin{IDS} να αλλάξει βάθος 5 φορές, δηλαδή θα τρέξει \textlatin{DLS} 5 φορές, κάθε μία με διαφορετικό όριο βάθους. Για το πρώτο \textlatin{DLS} 1 κόμβο, για το 2ο (1+3) κόμβους, για το 3ο \textlatin{DLS} (1+3+9) κόμβους, για το 4ο \textlatin{DLS} (1+3+9+27) κόμβους, για το 5ο 5 κόμβους.
Άρα στην καλύτερη περίπτωση θα αναπτύξει 63 κόμβους. Στην χειρότερη περίπτωση, ο κόμβος στόχος θα βρίσκεται στο δεξιότερο σημείο του 4ου επιπέδου. Άρα στο τελευταίο \textlatin{DLS} του θα αναπτύξει (1+3+9+27+81) κόμβους. Αυτό μαζί με τα προηγούμενα \textlatin{DLS} μας δίνει 179 αναπτυγμένους κόμβους. Βλέπουμε εδώ πάλι ότι δεν κάνει διαφορά η περαρότητα ή μη του χώρου, αφού ο \textlatin{IDS} έχει χαρακτηριστικά του \textlatin{BFS}, χάρη στα οποία αποφεύγει να αναπτύξει άπειρους κόμβους.
    \end{itemize}

\end{itemize}

\section*{Πρόβλημα 3:}
Θεωρήστε το παρακάτω πρόβλημα αναζήτησης. Ένα ρομπότ έχει αναλάβει να παραδώσει ένα δέμα από μια αποθήκη σε ένα προκαθορισμένο σημείο παράδοσης σε μια πόλη της οποίας ο χάρτης αναπαρίσταται ως 2D πλέγμα. Το ρομπότ μπορεί να κινηθεί προς τέσσερις κατευθύνσεις: πάνω, κάτω, αριστερά και δεξιά. Κάθε κίνηση έχει ένα κόστος που εξαρτάται από τον τύπο του εδάφους ή του δρόμου που πρέπει να διασχίσει το ρομπότ. Στόχος σας είναι να βοηθήσετε το ρομπότ να βρει τη βέλτιστη διαδρομή από την αποθήκη στο σημείο παράδοσης χρησιμοποιώντας τον αλγόριθμο A*.





\\ \\ \\ 
Το πλέγμα της πόλης περιέχει διάφορους τύπους εδάφους:
\begin{itemize}
    \item \textlatin{S}: Αφετηρία (Αποθήκη)
    \item \textlatin{G}: Στόχος (Σημείο παράδοσης)
    \item \textlatin{R}: Κανονικός δρόμος (κόστος 1)
    \item \textlatin{H}: Αυτοκινητόδρομος (κόστος 0.5)
    \item \textlatin{B}: Κτίριο (αδιάβατο εμπόδιο)
    \item \textlatin{P}: Πάρκο (κόστος 2)
    \item \textlatin{W}: Νερό (αδιάβατο εμπόδιο)
\end{itemize}

Το 10\textlatin{x}10 πλέγμα είναι το εξής:
\[
\begin{matrix}
S & R & R & R & B & W & R & H & H & H \\
R & B & B & R & H & H & R & R & B & H \\
R & P & P & R & B & R & R & R & B & R \\
R & R & R & R & W & R & P & P & R & R \\
R & R & B & R & R & R & H & H & R & B \\
B & W & R & P & P & R & B & R & R & R \\
P & P & R & R & R & R & R & R & B & B \\
R & B & R & R & R & W & H & H & R & R \\
R & R & R & R & B & R & R & R & B & R \\
H & H & H & B & B & R & R & G & R & R \\
\end{matrix}
\]

Το ρομπότ πρέπει να βρει τη διαδρομή από την αποθήκη (S) στο σημείο παράδοσης (G) που ελαχιστοποιεί το συνολικό κόστος μετακίνησης.

Να υπολογίσετε το συνολικό κόστος της βέλτιστης διαδρομής (άθροισμα του κόστους των κελιών που διασχίζονται) που θα βρεθεί από τον A* αν τον χρησιμοποιήσουμε στο πρόβλημα με ευρετική συνάρτηση την απόσταση \textlatin{Manhattan} διά 2. Δηλαδή το αποτέλεσμα της ευρετικής για τη μετάβαση από έναν κόμβο \textlatin{x} σε έναν κόμβο \textlatin{y}, δίνεται από τον παρακάτω τύπο:
\[
h(x, y) = \frac{\textlatin{ManhattanDistance}(x, y)}{2}
\] \\
\\ \\



Επίσης να υπολογίσετε:
\begin{itemize}
    \item Τον αριθμό των κόμβων που επεκτάθηκαν από τον A* κατά τη διάρκεια της αναζήτησης.
    \item Τη σειρά με την οποία βγαίνουν οι κόμβοι από τη λίστα «σύνορο» (\textlatin{fringe}).
\end{itemize}

Να δώσετε την ενέργεια του ρομπότ που αντιστοιχεί σε κάθε κόμβο.

Να υποθέσετε ότι:
\begin{itemize}
    \item (α) οι διαθέσιμες ενέργειες του ρομπότ εφαρμόζονται με τη σειρά που δόθηκαν παραπάνω, και 
    \item (β) όταν ο A* δεν μπορεί να «διακρίνει» δύο κόμβους, τότε επιλέγει τον αριστερότερο στο δένδρο αναζήτησης.
\end{itemize}

Τέλος, να προτείνετε δύο επιπλέον παραδεκτές ευρετικές συναρτήσεις για τη λύση του παραπάνω προβλήματος χρησιμοποιώντας τον αλγόριθμο A*. (Δεν χρειάζεται να λύσετε το πρόβλημα χρησιμοποιώντας τις, μόνο να τις ορίσετε και να αποδείξετε ότι είναι παραδεκτές.)

\section*{Απάντηση:}
Για την αναπαράσταση της λύσης θα κάνουμε τις εξής παραδοχές: κάθε κόμβος αναπαρίσταται σαν \textlatin{tuple ( x , y )} , για κάθε άξονα. Στο πρόβλημα μας \textlatin{x} είναι ο κάθετος άξονας και \textlatin{y} ο παράλληλος στο επίπεδο άξονας. Η αρίθμιση ξεκιννάει από το 0. Για την λύση μας θα έχουμε δύο δομές ένα \textlatin{PriorityQueue} και μια \textlatin{list}. Το \textlatin{PriorityQueue} θα λέγεται \textlatin{fringe} και θα περιλαμβάνει μόνο \textlatin{tuple} μορφής \textlatin{(x,y real\_cost, heuristic\_cost , move)}. Η λίστα θα περιλαμβάνει μια συλλογή από \textlatin{tuples}, όπου κάθε \textlatin{tuple} περιέχει τις συντεταγμένες \textlatin{x , y} , το πραγματικό ελάχιστο κόστος για να φτάσουμε σε αυτόν τον κόμβο, καθώς και την κίνηση που χρειάστηκε από τον πατέρα για να φτάσουμε σε αυτόν τον κόμβο. Συνεπώς θα είναι της μορφής \textlatin{ (x , y , real cost , move)}. Επίσης υποθέτω ότι το κόστος για να πάω στον κόμβο στόχο είναι 1. \\


\begin{itemize}
    \item 1o βήμα \\ \textlatin{Fringe: [( 0 , 0 , 0 , 8, 0)]} \\
           \textlatin{Explored: [ ]} 
\end{itemize}

\begin{itemize}
    \item 2o βήμα 
    \\ \textlatin{Fringe: [ ( 1,0,1,8.5,K), (0,1,1,8.5,D)]} \\
    \textlatin{Explored: [(0,0,0,0)]} 
\end{itemize}

\begin{itemize}
    \item 3o βήμα 
    \\ \textlatin{Fringe: [ (0,1,1,8.5,D)  (2,0,2,9,K)]} \\
    \textlatin{Explored: [(0,0,0,0) , ( 1,0,1,K) ]} 
\end{itemize}

\begin{itemize}
    \item 4o βήμα 
    \\ \textlatin{Fringe: [  (2,0,2,9,K) , (0,2,2,9,R) ]} \\
    \textlatin{Explored: [(0,0,0,0) , ( 1,0,1,K) , (0,1,1,D) ]} 
\end{itemize}

\begin{itemize}
    \item 5o βήμα 
    \\ \textlatin{Fringe: [   (0,2,2,9,D) , (3,0,3,9.5,K) , (2,1,4,10.5,D) ]} \\
    \textlatin{Explored: [(0,0,0,0) , ( 1,0,1,K) , (0,1,1,D) , (2,0,2,K) ]} 
\end{itemize}

\begin{itemize}
    \item 6o βήμα 
    \\ \textlatin{Fringe: [   (3,0,3,9.5,K) , (0,3,3,9.5,K) , (2,1,4,10.5,D) ]} \\
    \textlatin{Explored: [(0,0,0,0) , ( 1,0,1,K) , (0,1,1,D) , (2,0,2,K) , (0,2,2,D)]} 
\end{itemize}

\begin{itemize}
    \item 7o βήμα 
    \\ \textlatin{Fringe: [  (0,3,3,9.5,K) ,(4,0,4,10,K) , (3,1,4,10,D) (2,1,4,10.5,D) ]} \\
    \textlatin{Explored: [(0,0,0,0) , ( 1,0,1,K) , (0,1,1,D) , (2,0,2,K) , (0,2,2,D) , (3,0,3,K) ]} 
\end{itemize}

\begin{itemize}
    \item 8o βήμα 
    \\ \textlatin{Fringe: [  (4,0,4,10,K) , (3,1,4,10,D), (1,3,4,10,K),(2,1,4,10.5,D) ]} \\
    \textlatin{Explored: [(0,0,0,0) , ( 1,0,1,K) , (0,1,1,D) , (2,0,2,K) , (0,2,2,D) , (3,0,3,K) , (0,3,3,K)]} 
\end{itemize}


\begin{itemize}
    \item 9o βήμα 
    \\ \textlatin{Fringe: [  (3,1,4,10,D), (1,3,4,10,K),(2,1,4,10.5,D) , (4,1,5,10.5,D) ]} \\
    \textlatin{Explored: [(0,0,0,0) , ( 1,0,1,K) , (0,1,1,D) , (2,0,2,K) , (0,2,2,D) , (3,0,3,K) , (0,3,3,K) , (4,0,4,K)]} 
\end{itemize}

\begin{itemize}
    \item 10o βήμα 
    \\ \textlatin{Fringe: [ (1,3,4,10,K),(2,1,4,10.5,D) , (4,1,5,10.5,D) ,(3,2,5,10.5,D) ]} \\
    \textlatin{Explored: [(0,0,0,0) , ( 1,0,1,K) , (0,1,1,D) , (2,0,2,K) , (0,2,2,D) , (3,0,3,K) , (0,3,3,K) , (4,0,4,K) , (3,1,4,D)]} 
\end{itemize}

\begin{itemize}
    \item 11o βήμα 
    \\ \textlatin{Fringe: [ (1,4,4.5,10,K),(2,1,4,10.5,D) , (4,1,5,10.5,D) ,(3,2,5,10.5,D) ]} \\
    \textlatin{Explored: [(0,0,0,0) , ( 1,0,1,K) , (0,1,1,D) , (2,0,2,K) , (0,2,2,D) , (3,0,3,K) , (0,3,3,K) , (4,0,4,K) , (3,1,4,D) , (1,3,4,K)]} 
\end{itemize}

\begin{itemize}
    \item 12o βήμα 
    \\ \textlatin{Fringe: [ (1,5,5,10,K),(2,1,4,10.5,D) , (4,1,5,10.5,D) ,(3,2,5,10.5,D) ]} \\
    \textlatin{Explored: [(0,0,0,0) , ( 1,0,1,K) , (0,1,1,D) , (2,0,2,K) , (0,2,2,D) , (3,0,3,K) , (0,3,3,K) , (4,0,4,K) , (3,1,4,D) , (1,3,4,K), (1,4,4.5,K) ]} 
\end{itemize}

\begin{itemize}
    \item 13o βήμα 
    \\ \textlatin{Fringe: [ (2,1,4,10.5,D) , (4,1,5,10.5,D) ,(3,2,5,10.5,D), (2,5,6,10.5,K) , (1,6,6,10.5,D) ]} \\
    \textlatin{Explored: [(0,0,0,0) , ( 1,0,1,K) , (0,1,1,D) , (2,0,2,K) , (0,2,2,D) , (3,0,3,K) , (0,3,3,K) , (4,0,4,K) , (3,1,4,D) , (1,3,4,K), (1,4,4.5,K) , (1,5,5,K) ]} 
\end{itemize}

\begin{itemize}
    \item 14o βήμα 
    \\ \textlatin{Fringe: [ (4,1,5,10.5,D) ,(3,2,5,10.5,D), (2,5,6,10.5,K) , (1,6,6,10.5,D),
    (2,2,6,12,D)]
    } \\
    \textlatin{Explored: [(0,0,0,0) , ( 1,0,1,K) , (0,1,1,D) , (2,0,2,K) , (0,2,2,D) , (3,0,3,K) , (0,3,3,K) , (4,0,4,K) , (3,1,4,D) , (1,3,4,K), (1,4,4.5,K) , (1,5,5,K) , (2,1,4,D)]} 
\end{itemize}

\begin{itemize}
    \item 15o βήμα 
    \\ \textlatin{Fringe: [(3,2,5,10.5,D), (2,3,5,10.5 ,K), (2,5,6,10.5,K) , (1,6,6,10.5,D),
    (2,2,6,12,D)]
    } \\
    \textlatin{Explored: [(0,0,0,0) , ( 1,0,1,K) , (0,1,1,D) , (2,0,2,K) , (0,2,2,D) , (3,0,3,K) , (0,3,3,K) , (4,0,4,K) , (3,1,4,D) , (1,3,4,K), (1,4,4.5,K) , (1,5,5,K) , (2,1,4,D) , (4,1,5,D)]} 
\end{itemize}

\begin{itemize}
    \item 16o βήμα 
    \\ \textlatin{Fringe: [ (2,3,5,10.5 ,K), (2,5,6,10.5,K) , (1,6,6,10.5,D), (3,3,6,11,D)
    (2,2,6,12,D)]
    } \\
    \textlatin{Explored: [(0,0,0,0) , ( 1,0,1,K) , (0,1,1,D) , (2,0,2,K) , (0,2,2,D) , (3,0,3,K) , (0,3,3,K) , (4,0,4,K) , (3,1,4,D) , (1,3,4,K), (1,4,4.5,K) , (1,5,5,K) , (2,1,4,D) , (4,1,5,D) , (3,2,5,D)]} 
\end{itemize}

\begin{itemize}
    \item 17o βήμα 
    \\ \textlatin{Fringe: [(2,5,6,10.5,K) , (1,6,6,10.5,D), (3,3,6,11,D)
    (2,2,6,12,D)]
    } \\
    \textlatin{Explored: [(0,0,0,0) , ( 1,0,1,K) , (0,1,1,D) , (2,0,2,K) , (0,2,2,D) , (3,0,3,K) , (0,3,3,K) , (4,0,4,K) , (3,1,4,D) , (1,3,4,K), (1,4,4.5,K) , (1,5,5,K) , (2,1,4,D) , (4,1,5,D) , (3,2,5,D) ,(2,3,5,K) ]} 
\end{itemize}

\begin{itemize}
    \item 18o βήμα 
    \\ \textlatin{Fringe: [(1,6,6,10.5,D), (3,3,6,11,D) , (3,5,7,11,K) , (2,6,7,11,D)
    (2,2,6,12,D)]
    } \\
    \textlatin{Explored: [(0,0,0,0) , ( 1,0,1,K) , (0,1,1,D) , (2,0,2,K) , (0,2,2,D) , (3,0,3,K) , (0,3,3,K) , (4,0,4,K) , (3,1,4,D) , (1,3,4,K), (1,4,4.5,K) , (1,5,5,K) , (2,1,4,D) , (4,1,5,D) , (3,2,5,D) ,(2,3,5,K) , (2,5,6,K)  ]} 
\end{itemize}

\begin{itemize}
    \item 19o βήμα 
    \\ \textlatin{Fringe: [ (3,3,6,11,D) , (3,5,7,11,K) , (2,6,7,11,D) , (1,7,7,11,D)
    (2,2,6,12,D) , (0,6,7,12,P)]
    } \\
    \textlatin{Explored: [(0,0,0,0) , ( 1,0,1,K) , (0,1,1,D) , (2,0,2,K) , (0,2,2,D) , (3,0,3,K) , (0,3,3,K) , (4,0,4,K) , (3,1,4,D) , (1,3,4,K), (1,4,4.5,K) , (1,5,5,K) , (2,1,4,D) , (4,1,5,D) , (3,2,5,D) ,(2,3,5,K) , (2,5,6,K) , (1,6,6,D)]} 
\end{itemize}

\begin{itemize}
    \item 20o βήμα 
    \\ \textlatin{Fringe: [ (3,5,7,11,K) , (2,6,7,11,D) , (1,7,7,11,D),(4,3,7,11.5,K)
    (2,2,6,12,D) , (0,6,7,12,P)]
    } \\
    \textlatin{Explored: [(0,0,0,0) , ( 1,0,1,K) , (0,1,1,D) , (2,0,2,K) , (0,2,2,D) , (3,0,3,K) , (0,3,3,K) , (4,0,4,K) , (3,1,4,D) , (1,3,4,K), (1,4,4.5,K) , (1,5,5,K) , (2,1,4,D) , (4,1,5,D) , (3,2,5,D) ,(2,3,5,K) , (2,5,6,K) , (1,6,6,D) , (3,3,6,D)]} 
\end{itemize}

\begin{itemize}
    \item 21o βήμα 
    \\ \textlatin{Fringe: [(2,6,7,11,D) , (1,7,7,11,D),(4,3,7,11.5,K),(4,5,8,11.5,K)
    (2,2,6,12,D) , (0,6,7,12,P)\\,(3,6,9,12.5,D)]
    } \\
    \textlatin{Explored: [(0,0,0,0) , ( 1,0,1,K) , (0,1,1,D) , (2,0,2,K) , (0,2,2,D) , (3,0,3,K) , (0,3,3,K) , (4,0,4,K) , (3,1,4,D) , (1,3,4,K), (1,4,4.5,K) , (1,5,5,K) , (2,1,4,D) , (4,1,5,D) , (3,2,5,D) ,(2,3,5,K) , (2,5,6,K) , (1,6,6,D) , (3,3,6,D) , (3,5,7,K)]} 
\end{itemize}

\begin{itemize}
    \item 22o βήμα 
    \\ \textlatin{Fringe: [(1,7,7,11,D),(4,3,7,11.5,K),(4,5,8,11.5,K),(2,7,8,11.5,D)
    (2,2,6,12,D) , (0,6,7,12,P)\\,(3,6,9,12.5,D)]
    } \\
    \textlatin{Explored: [(0,0,0,0) , ( 1,0,1,K) , (0,1,1,D) , (2,0,2,K) , (0,2,2,D) , (3,0,3,K) , (0,3,3,K) , (4,0,4,K) , (3,1,4,D) , (1,3,4,K), (1,4,4.5,K) , (1,5,5,K) , (2,1,4,D) , (4,1,5,D) , (3,2,5,D) ,(2,3,5,K) , (2,5,6,K) , (1,6,6,D) , (3,3,6,D) , (3,5,7,K) , (2,6,7,D) ]} 
\end{itemize}

\begin{itemize}
    \item 23o βήμα 
    \\ \textlatin{Fringe: [(4,3,7,11.5,K),(4,5,8,11.5,K),(2,7,8,11.5,D)
    (2,2,6,12,D) , (0,6,7,12,P),(0,7,7.5,12,P)\\,(3,6,9,12.5,D)]
    } \\
    \textlatin{Explored: [(0,0,0,0) , ( 1,0,1,K) , (0,1,1,D) , (2,0,2,K) , (0,2,2,D) , (3,0,3,K) , (0,3,3,K) , (4,0,4,K) , (3,1,4,D) , (1,3,4,K), (1,4,4.5,K) , (1,5,5,K) , (2,1,4,D) , (4,1,5,D) , (3,2,5,D) ,(2,3,5,K) , (2,5,6,K) , (1,6,6,D) , (3,3,6,D) , (3,5,7,K) , (2,6,7,D) , (1,7,7,D)]} 
\end{itemize}

\begin{itemize}
    \item 24o βήμα 
    \\ \textlatin{Fringe: [(4,5,8,11.5,K),(2,7,8,11.5,D)
    (2,2,6,12,D) , (0,6,7,12,P),(0,7,7.5,12,P),(4,4,8,12,D),(3,6,9,12.5,D)\\,(5,3,9,13,K)]
    } \\
    \textlatin{Explored: [(0,0,0,0) , ( 1,0,1,K) , (0,1,1,D) , (2,0,2,K) , (0,2,2,D) , (3,0,3,K) , (0,3,3,K) , (4,0,4,K) , (3,1,4,D) , (1,3,4,K), (1,4,4.5,K) , (1,5,5,K) , (2,1,4,D) , (4,1,5,D) , (3,2,5,D) ,(2,3,5,K) , (2,5,6,K) , (1,6,6,D) , (3,3,6,D) , (3,5,7,K) , (2,6,7,D) , (1,7,7,D),(4,3,7,K)]} 
\end{itemize}

\begin{itemize}
    \item 25o βήμα 
    \\ \textlatin{Fringe: [(2,7,8,11.5,D),(4,6,8.5,11.5.D)
    (2,2,6,12,D) , (0,6,7,12,P),(0,7,7.5,12,P),(4,4,8,12,D),(5,5,9,12,K)\\,(3,6,9,12.5,D),(5,3,9,13,K)]
    } \\
    \textlatin{Explored: [(0,0,0,0) , ( 1,0,1,K) , (0,1,1,D) , (2,0,2,K) , (0,2,2,D) , (3,0,3,K) , (0,3,3,K) , (4,0,4,K) , (3,1,4,D) , (1,3,4,K), (1,4,4.5,K) , (1,5,5,K) , (2,1,4,D) , (4,1,5,D) , (3,2,5,D) ,(2,3,5,K) , (2,5,6,K) , (1,6,6,D) , (3,3,6,D) , (3,5,7,K) , (2,6,7,D) , (1,7,7,D),(4,3,7,K),(4,5,8,K)]} 
\end{itemize}

\begin{itemize}
    \item 26o βήμα 
    \\ \textlatin{Fringe: [(4,6,8.5,11.5.D)
    (2,2,6,12,D) , (0,6,7,12,P),(0,7,7.5,12,P),(4,4,8,12,D),(5,5,9,12,K),(3,6,9,12.5,D)\\,(5,3,9,13,K),(3,7,10,13,K)]
    } \\
    \textlatin{Explored: [(0,0,0,0) , ( 1,0,1,K) , (0,1,1,D) , (2,0,2,K) , (0,2,2,D) , (3,0,3,K) , (0,3,3,K) , (4,0,4,K) , (3,1,4,D) , (1,3,4,K), (1,4,4.5,K) , (1,5,5,K) , (2,1,4,D) , (4,1,5,D) , (3,2,5,D) ,(2,3,5,K) , (2,5,6,K) , (1,6,6,D) , (3,3,6,D) , (3,5,7,K) , (2,6,7,D) , (1,7,7,D),(4,3,7,K),(4,5,8,K),(2,7,8,D)]} 
\end{itemize}

\begin{itemize}
    \item 27o βήμα 
    \\ \textlatin{Fringe: [(4,7,9,11.5,D)
    (2,2,6,12,D) , (0,6,7,12,P),(0,7,7.5,12,P),(4,4,8,12,D),(5,5,9,12,K),(3,6,9,12.5,D)\\,(5,3,9,13,K),(3,7,10,13,K)]
    } \\
    \textlatin{Explored: [(0,0,0,0) , ( 1,0,1,K) , (0,1,1,D) , (2,0,2,K) , (0,2,2,D) , (3,0,3,K) , (0,3,3,K) , (4,0,4,K) , (3,1,4,D) , (1,3,4,K), (1,4,4.5,K) , (1,5,5,K) , (2,1,4,D) , (4,1,5,D) , (3,2,5,D) ,(2,3,5,K) , (2,5,6,K) , (1,6,6,D) , (3,3,6,D) , (3,5,7,K) , (2,6,7,D) , (1,7,7,D),(4,3,7,K),(4,5,8,K),(2,7,8,D),(4,6,8.5,D)]} 
\end{itemize}

\begin{itemize}
    \item 28o βήμα 
    \\ \textlatin{Fringe: [
    (2,2,6,12,D) , (0,6,7,12,P),(0,7,7.5,12,P),(4,4,8,12,D),(5,5,9,12,K),(5,7,10,12,K),(3,6,9,12.5,D)\\,(5,3,9,13,K),(3,7,10,13,K),(4,8,10,13,D)]
    } \\
    \textlatin{Explored: [(0,0,0,0) , ( 1,0,1,K) , (0,1,1,D) , (2,0,2,K) , (0,2,2,D) , (3,0,3,K) , (0,3,3,K) , (4,0,4,K) , (3,1,4,D) , (1,3,4,K), (1,4,4.5,K) , (1,5,5,K) , (2,1,4,D) , (4,1,5,D) , (3,2,5,D) ,(2,3,5,K) , (2,5,6,K) , (1,6,6,D) , (3,3,6,D) , (3,5,7,K) , (2,6,7,D) , (1,7,7,D),(4,3,7,K),(4,5,8,K),(2,7,8,D),(4,6,8.5,D),(4,7,9,D)]} 
\end{itemize}

\begin{itemize}
    \item 29o βήμα 
    \\ \textlatin{Fringe: [
    (0,6,7,12,P),(0,7,7.5,12,P),(4,4,8,12,D),(5,5,9,12,K),(5,7,10,12,K)(3,6,9,12.5,D),(5,3,9,13,K)\\,(3,7,10,13,K),(4,8,10,13,D)]
    } \\
    \textlatin{Explored: [(0,0,0,0) , ( 1,0,1,K) , (0,1,1,D) , (2,0,2,K) , (0,2,2,D) , (3,0,3,K) , (0,3,3,K) , (4,0,4,K) , (3,1,4,D) , (1,3,4,K), (1,4,4.5,K) , (1,5,5,K) , (2,1,4,D) , (4,1,5,D) , (3,2,5,D) ,(2,3,5,K) , (2,5,6,K) , (1,6,6,D) , (3,3,6,D) , (3,5,7,K) , (2,6,7,D) , (1,7,7,D),(4,3,7,K),(4,5,8,K),(2,7,8,D),(4,6,8.5,D),(4,7,9,D),\\(2,2,6,D)]} 
\end{itemize}

\begin{itemize}
    \item 30o βήμα 
    \\ \textlatin{Fringe: [
    (0,7,7.5,12,P),(4,4,8,12,D),(5,5,9,12,K),(5,7,10,12,K)(3,6,9,12.5,D),(5,3,9,13,K),(3,7,10,13,K)\\,(4,8,10,13,D)]
    } \\
    \textlatin{Explored: [(0,0,0,0) , ( 1,0,1,K) , (0,1,1,D) , (2,0,2,K) , (0,2,2,D) , (3,0,3,K) , (0,3,3,K) , (4,0,4,K) , (3,1,4,D) , (1,3,4,K), (1,4,4.5,K) , (1,5,5,K) , (2,1,4,D) , (4,1,5,D) , (3,2,5,D) ,(2,3,5,K) , (2,5,6,K) , (1,6,6,D) , (3,3,6,D) , (3,5,7,K) , (2,6,7,D) , (1,7,7,D),(4,3,7,K),(4,5,8,K),(2,7,8,D),(4,6,8.5,D),(4,7,9,D),\\(2,2,6,D),(0,6,7,P)]} 
\end{itemize}

\begin{itemize}
    \item 31o βήμα 
    \\ \textlatin{Fringe: [
    (4,4,8,12,D),(5,5,9,12,K),(5,7,10,12,K)(3,6,9,12.5,D),(5,3,9,13,K),(3,7,10,13,K),(4,8,10,13,D)\\,(0,8,10,13,D)]
    } \\
    \textlatin{Explored: [(0,0,0,0) , ( 1,0,1,K) , (0,1,1,D) , (2,0,2,K) , (0,2,2,D) , (3,0,3,K) , (0,3,3,K) , (4,0,4,K) , (3,1,4,D) , (1,3,4,K), (1,4,4.5,K) , (1,5,5,K) , (2,1,4,D) , (4,1,5,D) , (3,2,5,D) ,(2,3,5,K) , (2,5,6,K) , (1,6,6,D) , (3,3,6,D) , (3,5,7,K) , (2,6,7,D) , (1,7,7,D),(4,3,7,K),(4,5,8,K),(2,7,8,D),(4,6,8.5,D),(4,7,9,D),\\(2,2,6,D),(0,6,7,P),(0,7,7.5,P)]} 
\end{itemize}

\begin{itemize}
    \item 32o βήμα 
    \\ \textlatin{Fringe: [
    (5,5,9,12,K),(5,7,10,12,K)(3,6,9,12.5,D),(5,3,9,13,K),(3,7,10,13,K),(4,8,10,13,D),(0,8,10,13,D)\\,(5,4,10.5,13.5,K)]
    } \\
    \textlatin{Explored: [(0,0,0,0) , ( 1,0,1,K) , (0,1,1,D) , (2,0,2,K) , (0,2,2,D) , (3,0,3,K) , (0,3,3,K) , (4,0,4,K) , (3,1,4,D) , (1,3,4,K), (1,4,4.5,K) , (1,5,5,K) , (2,1,4,D) , (4,1,5,D) , (3,2,5,D) ,(2,3,5,K) , (2,5,6,K) , (1,6,6,D) , (3,3,6,D) , (3,5,7,K) , (2,6,7,D) , (1,7,7,D),(4,3,7,K),(4,5,8,K),(2,7,8,D),(4,6,8.5,D),(4,7,9,D),\\(2,2,6,D),(0,6,7,P),(0,7,7.5,P),(4,4,8,D)]} 
\end{itemize}
\vspace{1}
\begin{itemize}
    \item 33o βήμα 
    \\ \textlatin{Fringe: [
    (5,7,10,12,K),(3,6,9,12.5,D),(6,5,10,12.5,K),(5,3,9,13,K),(3,7,10,13,K),(4,8,10,13,D),(0,8,10,13,D)\\,(5,4,10.5,13.5,K)]
    } \\
    \textlatin{Explored: [(0,0,0,0) , ( 1,0,1,K) , (0,1,1,D) , (2,0,2,K) , (0,2,2,D) , (3,0,3,K) , (0,3,3,K) , (4,0,4,K) , (3,1,4,D) , (1,3,4,K), (1,4,4.5,K) , (1,5,5,K) , (2,1,4,D) , (4,1,5,D) , (3,2,5,D) ,(2,3,5,K) , (2,5,6,K) , (1,6,6,D) , (3,3,6,D) , (3,5,7,K) , (2,6,7,D) , (1,7,7,D),(4,3,7,K),(4,5,8,K),(2,7,8,D),(4,6,8.5,D),(4,7,9,D),\\(2,2,6,D),(0,6,7,P),(0,7,7.5,P),(4,4,8,D),(5,5,9,K)]} 
\end{itemize}

\begin{itemize}
    \item 34o βήμα 
    \\ \textlatin{Fringe: [
    (3,6,9,12.5,D),(6,5,10,12.5,K),(6,7,11,12.5,K),(5,3,9,13,K),(3,7,10,13,K),(4,8,10,13,D),(0,8,10,13,D)\\,(5,4,10.5,13.5,K),(5,8,11,13.5,D)]
    } \\
    \textlatin{Explored: [(0,0,0,0) , ( 1,0,1,K) , (0,1,1,D) , (2,0,2,K) , (0,2,2,D) , (3,0,3,K) , (0,3,3,K) , (4,0,4,K) , (3,1,4,D) , (1,3,4,K), (1,4,4.5,K) , (1,5,5,K) , (2,1,4,D) , (4,1,5,D) , (3,2,5,D) ,(2,3,5,K) , (2,5,6,K) , (1,6,6,D) , (3,3,6,D) , (3,5,7,K) , (2,6,7,D) , (1,7,7,D),(4,3,7,K),(4,5,8,K),(2,7,8,D),(4,6,8.5,D),(4,7,9,D),\\(2,2,6,D),(0,6,7,P),(0,7,7.5,P),(4,4,8,D),(5,5,9,K),(5,7,10,K)]} 
\end{itemize}

\begin{itemize}
    \item 35o βήμα 
    \\ \textlatin{Fringe: [
    (6,5,10,12.5,K),(6,7,11,12.5,K),(5,3,9,13,K),(3,7,10,13,K),(4,8,10,13,D),(0,8,10,13,D),(5,4,10.5,13.5,K)\\,(5,8,11,13.5,D)]
    } \\
    \textlatin{Explored: [(0,0,0,0) , ( 1,0,1,K) , (0,1,1,D) , (2,0,2,K) , (0,2,2,D) , (3,0,3,K) , (0,3,3,K) , (4,0,4,K) , (3,1,4,D) , (1,3,4,K), (1,4,4.5,K) , (1,5,5,K) , (2,1,4,D) , (4,1,5,D) , (3,2,5,D) ,(2,3,5,K) , (2,5,6,K) , (1,6,6,D) , (3,3,6,D) , (3,5,7,K) , (2,6,7,D) , (1,7,7,D),(4,3,7,K),(4,5,8,K),(2,7,8,D),(4,6,8.5,D),(4,7,9,D),\\(2,2,6,D),(0,6,7,P),(0,7,7.5,P),(4,4,8,D),(5,5,9,K),(5,7,10,K),(3,6,9,D)]} 
\end{itemize}

\begin{itemize}
    \item 36o βήμα 
    \\ \textlatin{Fringe: [
    (6,7,11,12.5,K),(5,3,9,13,K),(3,7,10,13,K),(4,8,10,13,D),(0,8,10,13,D),(6,6,11,13,D),(5,4,10.5,13.5,K)\\,(5,8,11,13.5,D),(6,4,11,14,A)]
    } \\
    \textlatin{Explored: [(0,0,0,0) , ( 1,0,1,K) , (0,1,1,D) , (2,0,2,K) , (0,2,2,D) , (3,0,3,K) , (0,3,3,K) , (4,0,4,K) , (3,1,4,D) , (1,3,4,K), (1,4,4.5,K) , (1,5,5,K) , (2,1,4,D) , (4,1,5,D) , (3,2,5,D) ,(2,3,5,K) , (2,5,6,K) , (1,6,6,D) , (3,3,6,D) , (3,5,7,K) , (2,6,7,D) , (1,7,7,D),(4,3,7,K),(4,5,8,K),(2,7,8,D),(4,6,8.5,D),(4,7,9,D),\\(2,2,6,D),(0,6,7,P),(0,7,7.5,P),(4,4,8,D),(5,5,9,K),(5,7,10,K),(3,6,9,D),(6,5,10,K)]} 
\end{itemize}

\begin{itemize}
    \item 37o βήμα 
    \\ \textlatin{Fringe: [
    (7,7,11.5,12.5,K)(5,3,9,13,K),(3,7,10,13,K),(4,8,10,13,D),(0,8,10,13,D),(6,6,11,13,D),(5,4,10.5,13.5,K)\\,(5,8,11,13.5,D),(6,4,11,14,A)]
    } \\
    \textlatin{Explored: [(0,0,0,0) , ( 1,0,1,K) , (0,1,1,D) , (2,0,2,K) , (0,2,2,D) , (3,0,3,K) , (0,3,3,K) , (4,0,4,K) , (3,1,4,D) , (1,3,4,K), (1,4,4.5,K) , (1,5,5,K) , (2,1,4,D) , (4,1,5,D) , (3,2,5,D) ,(2,3,5,K) , (2,5,6,K) , (1,6,6,D) , (3,3,6,D) , (3,5,7,K) , (2,6,7,D) , (1,7,7,D),(4,3,7,K),(4,5,8,K),(2,7,8,D),(4,6,8.5,D),(4,7,9,D),\\(2,2,6,D),(0,6,7,P),(0,7,7.5,P),(4,4,8,D),(5,5,9,K),(5,7,10,K),(3,6,9,D),(6,5,10,K),(6,7,11,K)]} 
\end{itemize}

\begin{itemize}
    \item 38o βήμα 
    \\ \textlatin{Fringe: [
    (5,3,9,13,K),(3,7,10,13,K),(4,8,10,13,D),(0,8,10,13,D),(6,6,11,13,D),(8,7,12.5,13,K),(5,4,10.5,13.5,K)\\,(5,8,11,13.5,D),(7,6,12,13.5,A),(6,4,11,14,A),(7,8,12.5,14,D)]
    } \\
    \textlatin{Explored: [(0,0,0,0) , ( 1,0,1,K) , (0,1,1,D) , (2,0,2,K) , (0,2,2,D) , (3,0,3,K) , (0,3,3,K) , (4,0,4,K) , (3,1,4,D) , (1,3,4,K), (1,4,4.5,K) , (1,5,5,K) , (2,1,4,D) , (4,1,5,D) , (3,2,5,D) ,(2,3,5,K) , (2,5,6,K) , (1,6,6,D) , (3,3,6,D) , (3,5,7,K) , (2,6,7,D) , (1,7,7,D),(4,3,7,K),(4,5,8,K),(2,7,8,D),(4,6,8.5,D),(4,7,9,D),\\(2,2,6,D),(0,6,7,P),(0,7,7.5,P),(4,4,8,D),(5,5,9,K),(5,7,10,K),(3,6,9,D),(6,5,10,K),(6,7,11,K),(7,7,11.5,K)]} 
\end{itemize}

\begin{itemize}
    \item 39o βήμα 
    \\ \textlatin{Fringe: [
    (3,7,10,13,K),(4,8,10,13,D),(0,8,10,13,D),(6,6,11,13,D),(8,7,12.5,13,K),(5,4,10.5,13.5,K),(5,8,11,13.5,D)\\,(7,6,12,13.5,A),(6,3,10.5,13.5,K),(6,4,11,14,A),(7,8,12.5,14,D),(5,2,10,14.5,A)]
    } \\
    \textlatin{Explored: [(0,0,0,0) , ( 1,0,1,K) , (0,1,1,D) , (2,0,2,K) , (0,2,2,D) , (3,0,3,K) , (0,3,3,K) , (4,0,4,K) , (3,1,4,D) , (1,3,4,K), (1,4,4.5,K) , (1,5,5,K) , (2,1,4,D) , (4,1,5,D) , (3,2,5,D) ,(2,3,5,K) , (2,5,6,K) , (1,6,6,D) , (3,3,6,D) , (3,5,7,K) , (2,6,7,D) , (1,7,7,D),(4,3,7,K),(4,5,8,K),(2,7,8,D),(4,6,8.5,D),(4,7,9,D),\\(2,2,6,D),(0,6,7,P),(0,7,7.5,P),(4,4,8,D),(5,5,9,K),(5,7,10,K),(3,6,9,D),(6,5,10,K),(6,7,11,K),(7,7,11.5,K),\\(5,3,9,K)]} 
\end{itemize}

\begin{itemize}
    \item 40o βήμα 
    \\ \textlatin{Fringe: [
    (4,8,10,13,D),(0,8,10,13,D),(6,6,11,13,D),(8,7,12.5,13,K),(5,4,10.5,13.5,K),(5,8,11,13.5,D),(7,6,12,13.5,A)\\,(6,3,10.5,13.5,K),(6,4,11,14,A),(7,8,12.5,14,D),(5,2,10,14.5,A),(3,8,11,14.5,D)]
    } \\
    \textlatin{Explored: [(0,0,0,0) , ( 1,0,1,K) , (0,1,1,D) , (2,0,2,K) , (0,2,2,D) , (3,0,3,K) , (0,3,3,K) , (4,0,4,K) , (3,1,4,D) , (1,3,4,K), (1,4,4.5,K) , (1,5,5,K) , (2,1,4,D) , (4,1,5,D) , (3,2,5,D) ,(2,3,5,K) , (2,5,6,K) , (1,6,6,D) , (3,3,6,D) , (3,5,7,K) , (2,6,7,D) , (1,7,7,D),(4,3,7,K),(4,5,8,K),(2,7,8,D),(4,6,8.5,D),(4,7,9,D),\\(2,2,6,D),(0,6,7,P),(0,7,7.5,P),(4,4,8,D),(5,5,9,K),(5,7,10,K),(3,6,9,D),(6,5,10,K),(6,7,11,K),(7,7,11.5,K),\\(5,3,9,K), (3,7,10,K)]} 
\end{itemize}


\begin{itemize}
    \item 41o βήμα 
    \\ \textlatin{Fringe: [
    (0,8,10,13,D),(6,6,11,13,D),(8,7,12.5,13,K),(5,4,10.5,13.5,K),(5,8,11,13.5,D),(7,6,12,13.5,A)\\,(6,3,10.5,13.5,K),(6,4,11,14,A),(7,8,12.5,14,D),(5,2,10,14.5,A),(3,8,11,14.5,D)]
    } \\
    \textlatin{Explored: [(0,0,0,0) , ( 1,0,1,K) , (0,1,1,D) , (2,0,2,K) , (0,2,2,D) , (3,0,3,K) , (0,3,3,K) , (4,0,4,K) , (3,1,4,D) , (1,3,4,K), (1,4,4.5,K) , (1,5,5,K) , (2,1,4,D) , (4,1,5,D) , (3,2,5,D) ,(2,3,5,K) , (2,5,6,K) , (1,6,6,D) , (3,3,6,D) , (3,5,7,K) , (2,6,7,D) , (1,7,7,D),(4,3,7,K),(4,5,8,K),(2,7,8,D),(4,6,8.5,D),(4,7,9,D),\\(2,2,6,D),(0,6,7,P),(0,7,7.5,P),(4,4,8,D),(5,5,9,K),(5,7,10,K),(3,6,9,D),(6,5,10,K),(6,7,11,K),(7,7,11.5,K),\\(5,3,9,K), (3,7,10,K),(4,8,10,D)]} 
\end{itemize}

\begin{itemize}
    \item 42o βήμα 
    \\ \textlatin{Fringe: [
    (6,6,11,13,D),(8,7,12.5,13,K),(5,4,10.5,13.5,K),(5,8,11,13.5,D),(7,6,12,13.5,A),(6,3,10.5,13.5,K)\\,(6,4,11,14,A),(7,8,12.5,14,D),(0,9,8.5,14,D),(5,2,10,14.5,A),(3,8,11,14.5,D)]
    } \\
    \textlatin{Explored: [(0,0,0,0) , ( 1,0,1,K) , (0,1,1,D) , (2,0,2,K) , (0,2,2,D) , (3,0,3,K) , (0,3,3,K) , (4,0,4,K) , (3,1,4,D) , (1,3,4,K), (1,4,4.5,K) , (1,5,5,K) , (2,1,4,D) , (4,1,5,D) , (3,2,5,D) ,(2,3,5,K) , (2,5,6,K) , (1,6,6,D) , (3,3,6,D) , (3,5,7,K) , (2,6,7,D) , (1,7,7,D),(4,3,7,K),(4,5,8,K),(2,7,8,D),(4,6,8.5,D),(4,7,9,D),\\(2,2,6,D),(0,6,7,P),(0,7,7.5,P),(4,4,8,D),(5,5,9,K),(5,7,10,K),(3,6,9,D),(6,5,10,K),(6,7,11,K),(7,7,11.5,K),\\(5,3,9,K), (3,7,10,K),(4,8,10,D),
    (0,8,10,D)]} 
\end{itemize}

\begin{itemize}
    \item 43o βήμα 
    \\ \textlatin{Fringe: [
    (8,7,12.5,13,K),(7,6,11.5,13,K),(5,4,10.5,13.5,K),(5,8,11,13.5,D),(6,3,10.5,13.5,K),(6,4,11,14,A)\\,(7,8,12.5,14,D),(0,9,8.5,14,D),(5,2,10,14.5,A),(3,8,11,14.5,D)]
    } \\
    \textlatin{Explored: [(0,0,0,0) , ( 1,0,1,K) , (0,1,1,D) , (2,0,2,K) , (0,2,2,D) , (3,0,3,K) , (0,3,3,K) , (4,0,4,K) , (3,1,4,D) , (1,3,4,K), (1,4,4.5,K) , (1,5,5,K) , (2,1,4,D) , (4,1,5,D) , (3,2,5,D) ,(2,3,5,K) , (2,5,6,K) , (1,6,6,D) , (3,3,6,D) , (3,5,7,K) , (2,6,7,D) , (1,7,7,D),(4,3,7,K),(4,5,8,K),(2,7,8,D),(4,6,8.5,D),(4,7,9,D),\\(2,2,6,D),(0,6,7,P),(0,7,7.5,P),(4,4,8,D),(5,5,9,K),(5,7,10,K),(3,6,9,D),(6,5,10,K),(6,7,11,K),(7,7,11.5,K),\\(5,3,9,K), (3,7,10,K),(4,8,10,D),
    (0,8,10,D),(6,6,11,D)]} 
\end{itemize}

\begin{itemize}
    \item 44o βήμα 
    \\ \textlatin{Fringe: [
    (7,6,11.5,13,K),(5,4,10.5,13.5,K),(5,8,11,13.5,D),(6,3,10.5,13.5,K),(9,7,13.5,13.5,K),(6,4,11,14,A)\\,(7,8,12.5,14,D),(0,9,8.5,14,D),(5,2,10,14.5,A),(3,8,11,14.5,D),(8,6,13.5,14.5,A)]
    } \\
    \textlatin{Explored: [(0,0,0,0) , ( 1,0,1,K) , (0,1,1,D) , (2,0,2,K) , (0,2,2,D) , (3,0,3,K) , (0,3,3,K) , (4,0,4,K) , (3,1,4,D) , (1,3,4,K), (1,4,4.5,K) , (1,5,5,K) , (2,1,4,D) , (4,1,5,D) , (3,2,5,D) ,(2,3,5,K) , (2,5,6,K) , (1,6,6,D) , (3,3,6,D) , (3,5,7,K) , (2,6,7,D) , (1,7,7,D),(4,3,7,K),(4,5,8,K),(2,7,8,D),(4,6,8.5,D),(4,7,9,D),\\(2,2,6,D),(0,6,7,P),(0,7,7.5,P),(4,4,8,D),(5,5,9,K),(5,7,10,K),(3,6,9,D),(6,5,10,K),(6,7,11,K),(7,7,11.5,K),\\(5,3,9,K), (3,7,10,K),(4,8,10,D),
    (0,8,10,D),(6,6,11,D), (8,7,12.5,K)]} 
\end{itemize}

\begin{itemize}
    \item 45o βήμα 
    \\ \textlatin{Fringe: [
    (5,4,10.5,13.5,K),(5,8,11,13.5,D),(6,3,10.5,13.5,K),(9,7,13.5,13.5,K),(8,6,12.5,13.5,A),(6,4,11,14,A)\\,(7,8,12.5,14,D),(0,9,8.5,14,D),(5,2,10,14.5,A),(3,8,11,14.5,D),]
    } \\
    \textlatin{Explored: [(0,0,0,0) , ( 1,0,1,K) , (0,1,1,D) , (2,0,2,K) , (0,2,2,D) , (3,0,3,K) , (0,3,3,K) , (4,0,4,K) , (3,1,4,D) , (1,3,4,K), (1,4,4.5,K) , (1,5,5,K) , (2,1,4,D) , (4,1,5,D) , (3,2,5,D) ,(2,3,5,K) , (2,5,6,K) , (1,6,6,D) , (3,3,6,D) , (3,5,7,K) , (2,6,7,D) , (1,7,7,D),(4,3,7,K),(4,5,8,K),(2,7,8,D),(4,6,8.5,D),(4,7,9,D),\\(2,2,6,D),(0,6,7,P),(0,7,7.5,P),(4,4,8,D),(5,5,9,K),(5,7,10,K),(3,6,9,D),(6,5,10,K),(6,7,11,K),(7,7,11.5,K),\\(5,3,9,K), (3,7,10,K),(4,8,10,D),
    (0,8,10,D),(6,6,11,D), (8,7,12.5,K),(7,6,11.5,K)]} 
\end{itemize}

\begin{itemize}
    \item 46o βήμα 
    \\ \textlatin{Fringe: [
    (5,8,11,13.5,D),(6,3,10.5,13.5,K),(9,7,13.5,13.5,K),(8,6,12.5,13.5,A),(6,4,11,14,A),(7,8,12.5,14,D)\\,(0,9,8.5,14,D),(5,2,10,14.5,A),(3,8,11,14.5,D),]
    } \\
    \textlatin{Explored: [(0,0,0,0) , ( 1,0,1,K) , (0,1,1,D) , (2,0,2,K) , (0,2,2,D) , (3,0,3,K) , (0,3,3,K) , (4,0,4,K) , (3,1,4,D) , (1,3,4,K), (1,4,4.5,K) , (1,5,5,K) , (2,1,4,D) , (4,1,5,D) , (3,2,5,D) ,(2,3,5,K) , (2,5,6,K) , (1,6,6,D) , (3,3,6,D) , (3,5,7,K) , (2,6,7,D) , (1,7,7,D),(4,3,7,K),(4,5,8,K),(2,7,8,D),(4,6,8.5,D),(4,7,9,D),\\(2,2,6,D),(0,6,7,P),(0,7,7.5,P),(4,4,8,D),(5,5,9,K),(5,7,10,K),(3,6,9,D),(6,5,10,K),(6,7,11,K),(7,7,11.5,K),\\(5,3,9,K), (3,7,10,K),(4,8,10,D),
    (0,8,10,D),(6,6,11,D), (8,7,12.5,K),(7,6,11.5,K),(5,4,10.5,K)]} 
\end{itemize}

\begin{itemize}
    \item 47o βήμα 
    \\ \textlatin{Fringe: [
    (6,3,10.5,13.5,K),(9,7,13.5,13.5,K),(8,6,12.5,13.5,A),(6,4,11,14,A),(7,8,12.5,14,D),(0,9,8.5,14,D)\\,(5,2,10,14.5,A),(3,8,11,14.5,D),(5,9,12,15,D)]
    } \\
    \textlatin{Explored: [(0,0,0,0) , ( 1,0,1,K) , (0,1,1,D) , (2,0,2,K) , (0,2,2,D) , (3,0,3,K) , (0,3,3,K) , (4,0,4,K) , (3,1,4,D) , (1,3,4,K), (1,4,4.5,K) , (1,5,5,K) , (2,1,4,D) , (4,1,5,D) , (3,2,5,D) ,(2,3,5,K) , (2,5,6,K) , (1,6,6,D) , (3,3,6,D) , (3,5,7,K) , (2,6,7,D) , (1,7,7,D),(4,3,7,K),(4,5,8,K),(2,7,8,D),(4,6,8.5,D),(4,7,9,D),\\(2,2,6,D),(0,6,7,P),(0,7,7.5,P),(4,4,8,D),(5,5,9,K),(5,7,10,K),(3,6,9,D),(6,5,10,K),(6,7,11,K),(7,7,11.5,K),\\(5,3,9,K), (3,7,10,K),(4,8,10,D),
    (0,8,10,D),(6,6,11,D), (8,7,12.5,K),(7,6,11.5,K),(5,4,10.5,K),(5,8,11,D)]} 
\end{itemize}

\begin{itemize}
    \item 48o βήμα 
    \\ \textlatin{Fringe: [
    (9,7,13.5,13.5,K),(8,6,12.5,13.5,A),(6,4,11,14,A),(7,8,12.5,14,D),(0,9,8.5,14,D),(7,3,11,14,K)\\,(5,2,10,14.5,A),(3,8,11,14.5,D),(5,9,12,15,D)]
    } \\
    \textlatin{Explored: [(0,0,0,0) , ( 1,0,1,K) , (0,1,1,D) , (2,0,2,K) , (0,2,2,D) , (3,0,3,K) , (0,3,3,K) , (4,0,4,K) , (3,1,4,D) , (1,3,4,K), (1,4,4.5,K) , (1,5,5,K) , (2,1,4,D) , (4,1,5,D) , (3,2,5,D) ,(2,3,5,K) , (2,5,6,K) , (1,6,6,D) , (3,3,6,D) , (3,5,7,K) , (2,6,7,D) , (1,7,7,D),(4,3,7,K),(4,5,8,K),(2,7,8,D),(4,6,8.5,D),(4,7,9,D),\\(2,2,6,D),(0,6,7,P),(0,7,7.5,P),(4,4,8,D),(5,5,9,K),(5,7,10,K),(3,6,9,D),(6,5,10,K),(6,7,11,K),(7,7,11.5,K),\\(5,3,9,K), (3,7,10,K),(4,8,10,D),
    (0,8,10,D),(6,6,11,D), (8,7,12.5,K),(7,6,11.5,K),(5,4,10.5,K),(5,8,11,D),\\(6,3,10.5,K)]} 
\end{itemize}

Στο τελευταίο βήμα απλά βγάζουμε από το fringe τον πρώτο κόμβο και βρήκαμε κατάσταση Goal, οπότε και τερματίζουμε τον αλγόριθμο.

\begin{itemize}
    \item 49o βήμα 
    \\ \textlatin{Fringe: [
    (8,6,12.5,13.5,A),(6,4,11,14,A),(7,8,12.5,14,D),(0,9,8.5,14,D),(7,3,11,14,K),(5,2,10,14.5,A)\\,(3,8,11,14.5,D),(5,9,12,15,D)]
    } \\
    \textlatin{Explored: [(0,0,0,0) , ( 1,0,1,K) , (0,1,1,D) , (2,0,2,K) , (0,2,2,D) , (3,0,3,K) , (0,3,3,K) , (4,0,4,K) , (3,1,4,D) , (1,3,4,K), (1,4,4.5,K) , (1,5,5,K) , (2,1,4,D) , (4,1,5,D) , (3,2,5,D) ,(2,3,5,K) , (2,5,6,K) , (1,6,6,D) , (3,3,6,D) , (3,5,7,K) , (2,6,7,D) , (1,7,7,D),(4,3,7,K),(4,5,8,K),(2,7,8,D),(4,6,8.5,D),(4,7,9,D),\\(2,2,6,D),(0,6,7,P),(0,7,7.5,P),(4,4,8,D),(5,5,9,K),(5,7,10,K),(3,6,9,D),(6,5,10,K),(6,7,11,K),(7,7,11.5,K),\\(5,3,9,K), (3,7,10,K),(4,8,10,D),
    (0,8,10,D),(6,6,11,D), (8,7,12.5,K),(7,6,11.5,K),(5,4,10.5,K),(5,8,11,D),\\(6,3,10.5,K),(9,7,13.5,K)]} 
\end{itemize}

Επομένως χρειάστηκε να κάνουμε \textlatin{expand} 47 κόμβους, μέχρις ότου το \textlatin{goal state} να έρθει ως πρώτο στοιχείο του \textlatin{fringe}. Επομένως το ελάχιστο κόστος για τον στόχο είναι 13.5 και επιτυγχάνεται με την διαδρομή : {[Δ, Δ, Δ, Κ, Δ, Δ, Κ, Κ, Κ, Δ, Δ, Κ, Κ, Κ, Κ, Κ]}.\\

Σε ότι αφορά το κομμάτι των ευρετικών, έχουμε 2 εναλλακτικές. Η πρώτη είναι απλά η ευκλείδια ευρετική που υπολογίζει την διαγώνια απόσταση ανάμεσα στο σημείο έναρξης και το σημείο εκκίνησης. Η συγκεκριμένη ευρετική είναι σίγουρα \textlatin{admissible}, αφού πάντα σε οποιοδήποτε ορθογώνιο τρίγωνο η υποτείνουσα είναι μικρότερη από το άθροισμα των προσκείμενων πλευρών του τριγώνου. Η δεύτερη ευρετική, αν και λιγότερο ρεαλιστική είναι να παίρνουμε την \textlatin{Manhattan Distance} για μία μόνο διάσταση αντί για τις δύο. Δηλαδή μόνο για τον άξονα \textlatin{x/y}.


\\ 
\section*{Πρόβλημα 4}

Θεωρήστε τον αλγόριθμο αμφίδρομης αναζήτησης που παρουσιάσαμε στις διαλέξεις.
Θεωρείστε ότι στα προβλήματα αναζήτησης που θα εφαρμοστεί ο αλγόριθμος
υπάρχει μοναδική κατάσταση στόχου. Υποθέτουμε ότι τα ζευγάρια αλγορίθμων που
χρησιμοποιεί η αμφίδρομη αναζήτηση σαν υπορουτίνες για την (προς τα εμπρός)
αναζήτηση από την αρχική κατάσταση και την (προς τα πίσω) αναζήτηση από την
κατάσταση στόχου είναι:

\begin{itemize}
    \item[(α)] Αναζήτηση πρώτα σε πλάτος και αναζήτηση περιορισμένου βάθους
    \item[(β)] Αναζήτηση με επαναληπτική εκβάθυνση και αναζήτηση περιορισμένου βάθους
    \item[(γ)] Α* και αναζήτηση περιορισμένου βάθους
    \item[(δ)] Α* και Α*
\end{itemize}

Είναι ο αλγόριθμος αμφίδρομης αναζήτησης με υπορουτίνες όπως στα (α)-(δ)
πλήρης; Είναι βέλτιστος; Ναι ή όχι και υπό ποιες συνθήκες. Πως μπορεί να γίνει
αποδοτικά ο έλεγχος ότι οι δύο αναζητήσεις συναντιούνται σε κάθε μια από τις
παραπάνω περιπτώσεις (α)-(δ);
\\

\section*{Απάντηση:}
\\
\begin{itemize}
    \item  Αναζήτηση πρώτα σε πλάτος και αναζήτηση περιορισμένου βάθους:
    

    \\ \\ Με βάση την συνάρτηση που μας δόθηκε στο φροντηστήριο ο αλγόριθμος θα αναπτύσσει συνέχεια κόμβους μέσω του \textlatin{DLS}.
    Αυτό συμβαίνει γιατί ο \textlatin{DLS} παράγει αρνητικές τιμές για το \textlatin{frontier} γιατί είναι τύπου \textlatin{DFS}. Εάν για τον \textlatin{DLS} επιλέξουμε πολύ κοντό βάθος τότε θα αναπτύσσει κόμβους μέχρι να φτάσει τα όρια του. Έπειτα θα αρχίσει να αναπτύσει κόμβους ο \textlatin{BFS} και συνεπώς θα βρεί μετά την λύση γιατί είναι πλήρης. Τώρα εάν για τον \textlatin{DLS} επιλέξουμε βάθος μεγαλύτερο από το βάθος της λύσης, τότε απλά ο \textlatin{DLS} θα αναπτύσσει συνέχεια κόμβους μέχρι να βρεί τον κόμβο στόχο. Επομένως αυτός ο συνδυασμός αλγορίθμων είναι πλήρης στην περίπτωση που ο παράγοντας διακλάδωσης είναι πεπερασμένος. Εάν δεν είναι τότε δεν θα βρεθεί ποτέ λύση.  Προφανώς και αυτός ο συνδυαασμός δεν παράγει βέλτιστη λύση γιατί δεν λαμβάνει υπόψη παραμέτρους όπως τα διαφορετικά κόστη κάποιον μονοπατιών.
    \end{itemize}
\vspace{1}

\begin{itemize}

    \item  Αναζήτηση με επαναληπτική εκβάθυνση και αναζήτηση περιορισμένου βάθους: 
    

    \\ \\ Με βάση πάλι την συνάρτηση του φροντηστηρίου και οι δύο αλγόριθμοι θα παράγουν αρνητικές τιμές για το \textlatin{frontier}, όμως επειδή στην αρχή ξεκινούν  με 0 θα καλείται συνέχεια  ο \textlatin{DLS}. Εάν επιλέξουμε πάλι κοντό βάθος θα εξερευνήσει μέχρι ένα σημείο και μετά θα ξεκινήσει να ψάχνει ο \textlatin{IDS} και αυξάνοντας σιγά σιγά το βάθος θα  καταφέρει να βρεί την λύση. Τώρα βάλουμε βάθος στον \textlatin{DLS} μεγαλύτερο από τον βάθος της διαδρομής στόχου,  τότε απλά θα αναπτύσσει πάλι κόμβους μέχρι να βρεί τον στόχο. Άρα πάλι σε γενικές γραμμές ο συνδυασμός είναι πλήρης. Εάν όμως ο παράγοντας διακλάδωσης δεν είναι πεπερασμένος δεν θα μπορέσουμε να βρούμε λύση. Ο συνδυασμός αυτός πάλι δεν μας  εγγυάται βέλτιστη λύση, αφού δεν τον ενδιαφέρουν τα κόστη των διαδρομών, αλλά οι ρηχές διαδρομές.

\end{itemize}
\vspace{2}



\begin{itemize}

    \item  \textlatin{A}* και αναζήτηση περιορισμένου βάθους:

    \\ \\ Πάλι σε  αυτή την περίπτωση για τους ίδιους λόγους  με παραπάνω θα αρχίσει να προχωράει ο \textlatin{DLS} πρώτος. Εάν πάρουμε κοντό βάθος θα πάει μέχρι ένα σημείο και  μετά την λύση θα την βρεί ο \textlatin{A*}. Ακόμη βάθος μεγαλύτερο τους βάθους λύσης να βάλουμε ο \textlatin{DLS} θα καταφέρει να βρεί λύση. Πάλι σε γενικές γραμμές θα είναι πλήρης ο συνδυασμός, αρκεί ο παράγοντας διακλάδωσης να είναι πεπερασμένος. Η λύση που θα παράγεται δεν  είναι  βέλτιστη. Μέχρι το σημείο που συναντιούνται οι αλγόριθμοι ο \textlatin{A*} ακολουθεί καλό μονοπάτι, αλλά από το σημείο και μετά που ακολουθούμε την διαδρομή που χάραξε ο \textlatin{DLS}, δεν μας  εγγυάται κανείς ότι πάμε με το βέλτιστο μονοπάτι.

\end{itemize}


\vspace{1}

\begin{itemize}

    \item  \textlatin{A}* και \textlatin{A}*:
    \\ \\
    Ο τελευταίος συνδυασμός είναι πλήρης προφανώς γιατί είναι και οι δύο \textlatin{A}*. Στην περίπτωση μη πεπερασμένου παράγοντα διακλάδωσης όμως παύει να είναι πλήρης. Πάντως σε  γενικές γραμμές με \textlatin{admissible} ευρετικές πάντα συναντιώονται και μάλιστα παράγουν την βέλτιστη λύση, γιατί λαμβάνουν υπόψη τα διαφορετικά κόστη κάθε κόμβου και κινούνται κάθε φορά προς την διαδρομή με το μικρότερο κόστος.

\end{itemize}
\\

Ο καλύτερος τρόπος για να ελέγχουμε αν οι δύο αναζητήσεις συναντιούνται αποδοτικά σε όλες τις παραπάνω περιπτώσεις, είναι για κάθε αλγόριθμο να δημιουργούμε ένα \textlatin{set} που υλοποιείται με \textlatin{hashmaps} και αποθηκεύει όλους τους κόμβους που έχει επισκεπτεί ο κάθε αλγόριθμος. Επειδή μπορούμε να ελέγχουμε αν ένας κόμβος είναι στο set σε Ο(1), αυτή η λύση θα είναι η καλύτερη δυνατή.

\end{document}
\end{document}




\end{document}


